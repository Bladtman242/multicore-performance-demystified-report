\chapter{Advice for multicore programmers}
\label{chap:advice}
We have taken a look at cache behaviours, and seen that extra care needs to
taken when writing cache-friendly multicore software.
In particular, the cache coherence protocol means that grouping data closely
together in memory, as we are wont to do in sequential software, can cause
devastating false sharing overhead on multicore systems.

We have tried to relate the concepts of cache coherence and memory barriers and
memory layout to Java, and have seen that steps can be taken to affect multicore
cache performance in Java programs significantly:

We have seen from experiments that different multicore designs using locks or
optimistic concurrency via compare-and-swap can be particularly sensitive to
false sharing, and that using padded or spaced memory layouts can significantly
improve performance of such applications. We have looked at designs that avoid
sharing data between CPU cores by using thread-local data structures as the
program runs, and only consolidating the results in end. When available, these
designs perform remarkably well: They limit not only false sharing, but data
sharing in general.

The lessons learned from this project are:
\begin{enumerate}
\item Sharing data between CPUs is expensive. So when practical, use
thread-local designs to avoid sharing.
\item When avoidance is not practical, identify fields and data structures that
are subject to updates, and use padding to ensure the updates do not cause false
invalidations. This applies to all data, but is particularly important to synchronization variables like
volatile fields, locks, and instances of the classes under the
\java{java.util.concurrent} package.
\item False invalidation gets worse as the number of cores increase, but can be
	highly significant with just two physical cores.
\item All updates are candidates for false sharing overhead: Even if only a
single thread performs writes, reading threads can force it to change a cache
line's state from exclusive to shared. This forces the writing thread to perform --
possibly false -- invalidations for later updates.
\item Study the hardware. Spending a half an hour reading optimization manuals for the hardware
platforms you use can replace hours of scouring the internet.
\item Benchmark multicore code. Studying hardware designs, while interesting
and useful, is a rabbit hole of unbounded depth. Designing software with the
cache coherence protocol in mind, we still ran into trouble with the L2 cache's
spatial prefetcher with most locking applications. Or at least we think we did.
We cannot know for certain, and that is exactly the point.
\end{enumerate}

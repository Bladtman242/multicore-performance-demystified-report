\begin{figure}[hbpt]
	\graphicspath{{plots/}}
	\begin{subfigure}{0.32\textwidth}
		\input{plots/Histo-lock-local-0-x1.tex}
		\caption{0 bytes}
	\end{subfigure}
	\begin{subfigure}{0.32\textwidth}
		\input{plots/Histo-lock-local-16-x1.tex}
		\caption{16 bytes}
	\end{subfigure}
	\begin{subfigure}{0.32\textwidth}
		\input{plots/Histo-lock-local-32-x1.tex}
		\caption{32 bytes}
	\end{subfigure}
	\begin{subfigure}{0.32\textwidth}
		\input{plots/Histo-lock-local-48-x1.tex}
		\caption{48 bytes}
	\end{subfigure}
	\begin{subfigure}{0.32\textwidth}
		\input{plots/Histo-lock-local-64-x1.tex}
		\caption{64 bytes}
	\end{subfigure}
	\begin{subfigure}{0.32\textwidth}
		\input{plots/Histo-lock-local-80-x1.tex}
		\caption{80 bytes}
	\end{subfigure}
	\begin{subfigure}{0.32\textwidth}
		\input{plots/Histo-lock-local-96-x1.tex}
		\caption{96 bytes}
	\end{subfigure}
	\begin{subfigure}{0.32\textwidth}
		\input{plots/Histo-lock-local-112-x1.tex}
		\caption{112 bytes}
	\end{subfigure}
	\begin{subfigure}{0.32\textwidth}
		\input{plots/Histo-lock-local-128-x1.tex}
		\caption{128 bytes}
	\end{subfigure}
	\caption{The histogram problem on the i5 platform. Here the impact of
	padding the locks is negative, while padding the buckets improves
	performance by 10.7\%. The best time is 100.88 milliseconds, achieved by
	padding the buckets with 128 bytes, and not padding the locks.
	Each plot uses a different amount of padding between the locks
	(specified beneath each plot). The plots show the running time of the
	histogram problem as a function of the amount of padding between the
	histogram buckets in bytes. Each y-tick is 50 milliseconds. The axis
	starts at 0.}
	\label{fig:histo-local-i5}
\end{figure}

\begin{figure}[hbpt]
	\graphicspath{{plots/}}
	\begin{subfigure}{0.32\textwidth}
		\input{plots/Histo-lock-local-0-itu-desktop.tex}
		\caption{0 bytes}
	\end{subfigure}
	\begin{subfigure}{0.32\textwidth}
		\input{plots/Histo-lock-local-16-itu-desktop.tex}
		\caption{16 bytes}
	\end{subfigure}
	\begin{subfigure}{0.32\textwidth}
		\input{plots/Histo-lock-local-32-itu-desktop.tex}
		\caption{32 bytes}
	\end{subfigure}
	\begin{subfigure}{0.32\textwidth}
		\input{plots/Histo-lock-local-48-itu-desktop.tex}
		\caption{48 bytes}
	\end{subfigure}
	\begin{subfigure}{0.32\textwidth}
		\input{plots/Histo-lock-local-64-itu-desktop.tex}
		\caption{64 bytes}
	\end{subfigure}
	\begin{subfigure}{0.32\textwidth}
		\input{plots/Histo-lock-local-80-itu-desktop.tex}
		\caption{80 bytes}
	\end{subfigure}
	\begin{subfigure}{0.32\textwidth}
		\input{plots/Histo-lock-local-96-itu-desktop.tex}
		\caption{96 bytes}
	\end{subfigure}
	\begin{subfigure}{0.32\textwidth}
		\input{plots/Histo-lock-local-112-itu-desktop.tex}
		\caption{112 bytes}
	\end{subfigure}
	\begin{subfigure}{0.32\textwidth}
		\input{plots/Histo-lock-local-128-itu-desktop.tex}
		\caption{128 bytes}
	\end{subfigure}
	\caption{The histogram problem on the i7 platform. Here the overall
	trend is that padding improves performance with respect to both buckets
	and locks. The best time is 59.9 milliseconds, achieved by padding the
	locks with 112 bytes, and the buckets with 128 bytes. That is a 16.9\%
	improvement on not using padding. Each plot uses a different amount of
	padding between the locks (specified beneath each plot). The plots show
	the running time of the histogram problem as a function of the amount of
	padding between the histogram buckets in bytes. Each y-tick is 10
	milliseconds. The axis starts at 0.} \label{fig:histo-local-i7}
\end{figure}

\begin{figure}[hbpt]
	\graphicspath{{plots/}}
	\begin{subfigure}{0.32\textwidth}
		\input{plots/Histo-lock-local-0-itu-server.tex}
		\caption{0 bytes}
	\end{subfigure}
	\begin{subfigure}{0.32\textwidth}
		\input{plots/Histo-lock-local-16-itu-server.tex}
		\caption{16 bytes}
	\end{subfigure}
	\begin{subfigure}{0.32\textwidth}
		\input{plots/Histo-lock-local-32-itu-server.tex}
		\caption{32 bytes}
	\end{subfigure}
	\begin{subfigure}{0.32\textwidth}
		\input{plots/Histo-lock-local-48-itu-server.tex}
		\caption{48 bytes}
	\end{subfigure}
	\begin{subfigure}{0.32\textwidth}
		\input{plots/Histo-lock-local-64-itu-server.tex}
		\caption{64 bytes}
	\end{subfigure}
	\begin{subfigure}{0.32\textwidth}
		\input{plots/Histo-lock-local-80-itu-server.tex}
		\caption{80 bytes}
	\end{subfigure}
	\begin{subfigure}{0.32\textwidth}
		\input{plots/Histo-lock-local-96-itu-server.tex}
		\caption{96 bytes}
	\end{subfigure}
	\begin{subfigure}{0.32\textwidth}
		\input{plots/Histo-lock-local-112-itu-server.tex}
		\caption{112 bytes}
	\end{subfigure}
	\begin{subfigure}{0.32\textwidth}
		\input{plots/Histo-lock-local-128-itu-server.tex}
		\caption{128 bytes}
	\end{subfigure}
	\caption{The histogram problem on the Xeon platform. Here padding has a
	large effect, both for the buckets and the locks. The best time is
	118.62 miliseconds, achieved by padding the locks with 112 bytes, and
	the histograms with 128 bytes. That is a 71.8\% improvement on not using
	padding.
	Each plot uses a
	different amount of padding between the locks (specified beneath each
	plot). The plots show the running time of the histogram problem as a
	function of the amount of padding between the histogram buckets in
	bytes. Each y-tick is 100 milliseconds. The axis starts at 0.}
	\label{fig:histo-local-xeon}
\end{figure}

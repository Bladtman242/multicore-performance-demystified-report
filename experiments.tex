\section{Experiments}
\mynote{Note high std dev where relevant}
\subsection{Multicore Cache Performance}
\mynote{\textbf{contents: (parrallelisability?)}}

To see the potential impact of false sharing, it is illustrative to look at a
few contrived example programs. In this section we look at a handful of
variations of programs that perform integer-increment operations in a multi core
setting.

\subsubsection{Uncontended Writes}
The first example we examine illustrates the impact of false sharing by running
simple, uncontended, integer-increment operations in parallel threads. To see
the impact of false sharing, we observe the time it takes to perform an
increment as a function of the distance between the integers in memory.

\mynote{add code excerpt around here}

Each thread performs millions of integer-increments. The integers are
uncontended; each thread has its own integer and performs no read or write
operations to integers used by the other threads. Since each thread operates on
its own integer, there is no need for synchronization. Nonetheless, we perform
the experiment in variants with and without volatile integer declarations, to
see the performance impact in both cases.

Since the integers are uncontended, we will take the difference in performance
to be a result of unnecessary coherence overhead due to false sharing.

In the experiments with volatile integers, each thread performs 6M increments.
In the experiments without volatile, each thread performs 66M increments. The
experiments are run with 4, 8, and 48 threads on the i5, i7, and xeon platforms,
respectively.

Figure \ref{fig:uncont} and \ref{fig:uncont-nob} show the average time per
increment.

\mynote{Include concrete parameters: work is not divided between threads, but is
additive, lap/desk/serv performs x,y,z ops..}

\begin{figure}[hbpt]
\graphicspath{{plots/}}
\input{plots/barrier.tex}
\caption{Uncontended increments on volatile integers. The plot shows nanoseconds
	per increment, as a function of the number of bytes used as padding
	between the integers.}
	\label{fig:uncont}
\end{figure}

\begin{figure}[hbpt]
\graphicspath{{plots/}}
\input{plots/nobarrier.tex}
\caption{Uncontended increments on non-volatile integers. The plot shows nanoseconds
	per increment, as a function of the number of bytes used as padding
	between the integers.}
\label{fig:uncont-nob}
\end{figure}

\mynote{Add explanation of the different plots (array/local,
barrier/no-barrier), include values of plots (plots are hard to read), and
reflect/conclude}

\subsubsection{Contended Writes}

In the previous section we examined the performance of uncontended memory
operations to see the existence and cost of false sharing. In this section, we
shall see that observing the performance of contended memory operations - where
coherence overhead is strictly necessary - can be just as illustrative.

\mynote{include code snippet}

The program is similar to \mynote{reference uncontended snippet}, but in this version all threads operate on a single, shared integer. Like in the previous
section, we run two versions of the experiment: One where the integer is
volatile, and one where it is not. As this experiment uses a shared integer,
there is no need to store it in an array. However, plots using a single-element
array are included, to show that the behaviour is not significantly affected by
this change.

An additional experiment is included here, running an additional thread that
only reads the integer. It allows us to see that, perhaps contrary to intuition,
read operations in a multicore setting also incur a coherency overhead.

\mynote{include getter/stop code snippet}

We need the reading thread to run for the full duration of the experiment. To
that end, the thread does not perform a predetermined number of operations, but
is started before the others, and terminated only when the others are
finished. This incurs an overhead, as the thread needs to be stopped before the
benchmark finishes! However, an additional experiment has shown that the time it
takes to stop a thread is far too small to significantly skew our results.

\mynote{include table of numbers (they are 833.4$\pm$2.82 (desktop),
1940.5$\pm$58.12 (server),
and 1256.7$\pm$13.59 (laptop) ns)})

\begin{figure}[hbpt]
	\graphicspath{{plots/}}
	\begin{subfigure}{0.5\textwidth}
		\input{plots/sharedintbarrier.tex}
		\caption{i5}
	\end{subfigure}
	\begin{subfigure}{0.5\textwidth}
		\input{plots/sharedintbarrierdesktop.tex}
		\caption{i7}
	\end{subfigure}
	\begin{subfigure}{1\textwidth}
		% GNUPLOT: LaTeX picture with Postscript
\begingroup
  \makeatletter
  \providecommand\color[2][]{%
    \GenericError{(gnuplot) \space\space\space\@spaces}{%
      Package color not loaded in conjunction with
      terminal option `colourtext'%
    }{See the gnuplot documentation for explanation.%
    }{Either use 'blacktext' in gnuplot or load the package
      color.sty in LaTeX.}%
    \renewcommand\color[2][]{}%
  }%
  \providecommand\includegraphics[2][]{%
    \GenericError{(gnuplot) \space\space\space\@spaces}{%
      Package graphicx or graphics not loaded%
    }{See the gnuplot documentation for explanation.%
    }{The gnuplot epslatex terminal needs graphicx.sty or graphics.sty.}%
    \renewcommand\includegraphics[2][]{}%
  }%
  \providecommand\rotatebox[2]{#2}%
  \@ifundefined{ifGPcolor}{%
    \newif\ifGPcolor
    \GPcolorfalse
  }{}%
  \@ifundefined{ifGPblacktext}{%
    \newif\ifGPblacktext
    \GPblacktexttrue
  }{}%
  % define a \g@addto@macro without @ in the name:
  \let\gplgaddtomacro\g@addto@macro
  % define empty templates for all commands taking text:
  \gdef\gplbacktext{}%
  \gdef\gplfronttext{}%
  \makeatother
  \ifGPblacktext
    % no textcolor at all
    \def\colorrgb#1{}%
    \def\colorgray#1{}%
  \else
    % gray or color?
    \ifGPcolor
      \def\colorrgb#1{\color[rgb]{#1}}%
      \def\colorgray#1{\color[gray]{#1}}%
      \expandafter\def\csname LTw\endcsname{\color{white}}%
      \expandafter\def\csname LTb\endcsname{\color{black}}%
      \expandafter\def\csname LTa\endcsname{\color{black}}%
      \expandafter\def\csname LT0\endcsname{\color[rgb]{1,0,0}}%
      \expandafter\def\csname LT1\endcsname{\color[rgb]{0,1,0}}%
      \expandafter\def\csname LT2\endcsname{\color[rgb]{0,0,1}}%
      \expandafter\def\csname LT3\endcsname{\color[rgb]{1,0,1}}%
      \expandafter\def\csname LT4\endcsname{\color[rgb]{0,1,1}}%
      \expandafter\def\csname LT5\endcsname{\color[rgb]{1,1,0}}%
      \expandafter\def\csname LT6\endcsname{\color[rgb]{0,0,0}}%
      \expandafter\def\csname LT7\endcsname{\color[rgb]{1,0.3,0}}%
      \expandafter\def\csname LT8\endcsname{\color[rgb]{0.5,0.5,0.5}}%
    \else
      % gray
      \def\colorrgb#1{\color{black}}%
      \def\colorgray#1{\color[gray]{#1}}%
      \expandafter\def\csname LTw\endcsname{\color{white}}%
      \expandafter\def\csname LTb\endcsname{\color{black}}%
      \expandafter\def\csname LTa\endcsname{\color{black}}%
      \expandafter\def\csname LT0\endcsname{\color{black}}%
      \expandafter\def\csname LT1\endcsname{\color{black}}%
      \expandafter\def\csname LT2\endcsname{\color{black}}%
      \expandafter\def\csname LT3\endcsname{\color{black}}%
      \expandafter\def\csname LT4\endcsname{\color{black}}%
      \expandafter\def\csname LT5\endcsname{\color{black}}%
      \expandafter\def\csname LT6\endcsname{\color{black}}%
      \expandafter\def\csname LT7\endcsname{\color{black}}%
      \expandafter\def\csname LT8\endcsname{\color{black}}%
    \fi
  \fi
    \setlength{\unitlength}{0.0500bp}%
    \ifx\gptboxheight\undefined%
      \newlength{\gptboxheight}%
      \newlength{\gptboxwidth}%
      \newsavebox{\gptboxtext}%
    \fi%
    \setlength{\fboxrule}{0.5pt}%
    \setlength{\fboxsep}{1pt}%
\begin{picture}(7200.00,2520.00)%
    \gplgaddtomacro\gplbacktext{%
      \csname LTb\endcsname%%
      \put(528,406){\makebox(0,0)[r]{\strut{}$0$}}%
      \put(528,616){\makebox(0,0)[r]{\strut{}$200$}}%
      \put(528,827){\makebox(0,0)[r]{\strut{}$400$}}%
      \put(528,1037){\makebox(0,0)[r]{\strut{}$600$}}%
      \put(528,1248){\makebox(0,0)[r]{\strut{}$800$}}%
      \put(528,1458){\makebox(0,0)[r]{\strut{}$1000$}}%
      \put(528,1669){\makebox(0,0)[r]{\strut{}$1200$}}%
      \put(528,1879){\makebox(0,0)[r]{\strut{}$1400$}}%
      \put(528,2090){\makebox(0,0)[r]{\strut{}$1600$}}%
      \put(528,2300){\makebox(0,0)[r]{\strut{}$1800$}}%
      \put(660,274){\rotatebox{-45}{\makebox(0,0)[l]{\strut{}1}}}%
      \put(753,274){\rotatebox{-45}{\makebox(0,0)[l]{\strut{}2}}}%
      \put(940,274){\rotatebox{-45}{\makebox(0,0)[l]{\strut{}4}}}%
      \put(1313,274){\rotatebox{-45}{\makebox(0,0)[l]{\strut{}8}}}%
      \put(1686,274){\rotatebox{-45}{\makebox(0,0)[l]{\strut{}12}}}%
      \put(2060,274){\rotatebox{-45}{\makebox(0,0)[l]{\strut{}16}}}%
      \put(2806,274){\rotatebox{-45}{\makebox(0,0)[l]{\strut{}24}}}%
      \put(3553,274){\rotatebox{-45}{\makebox(0,0)[l]{\strut{}32}}}%
      \put(5046,274){\rotatebox{-45}{\makebox(0,0)[l]{\strut{}48}}}%
      \put(6539,274){\rotatebox{-45}{\makebox(0,0)[l]{\strut{}64}}}%
    }%
    \gplgaddtomacro\gplfronttext{%
    }%
    \gplbacktext
    \put(0,0){\includegraphics{sharedintbarrierserver}}%
    \gplfronttext
  \end{picture}%
\endgroup

		\caption{xeon}
	\end{subfigure}
	\caption{Contended increments on \textbf{volatile} integers. The plots show
	nanoseconds per increment, as a function of the number of threads.
	Please Note that both axes vary across the plots.}
	\label{fig:cont}
\end{figure}

\begin{figure}[hbpt]
	\graphicspath{{plots/}}
	\begin{subfigure}{0.5\textwidth}
		\input{plots/sharedint.tex}
		\caption{i5}
	\end{subfigure}
	\begin{subfigure}{0.5\textwidth}
		\input{plots/sharedintdesktop.tex}
		\caption{i7}
	\end{subfigure}
	\begin{subfigure}{1\textwidth}
		% GNUPLOT: LaTeX picture with Postscript
\begingroup
  \makeatletter
  \providecommand\color[2][]{%
    \GenericError{(gnuplot) \space\space\space\@spaces}{%
      Package color not loaded in conjunction with
      terminal option `colourtext'%
    }{See the gnuplot documentation for explanation.%
    }{Either use 'blacktext' in gnuplot or load the package
      color.sty in LaTeX.}%
    \renewcommand\color[2][]{}%
  }%
  \providecommand\includegraphics[2][]{%
    \GenericError{(gnuplot) \space\space\space\@spaces}{%
      Package graphicx or graphics not loaded%
    }{See the gnuplot documentation for explanation.%
    }{The gnuplot epslatex terminal needs graphicx.sty or graphics.sty.}%
    \renewcommand\includegraphics[2][]{}%
  }%
  \providecommand\rotatebox[2]{#2}%
  \@ifundefined{ifGPcolor}{%
    \newif\ifGPcolor
    \GPcolorfalse
  }{}%
  \@ifundefined{ifGPblacktext}{%
    \newif\ifGPblacktext
    \GPblacktexttrue
  }{}%
  % define a \g@addto@macro without @ in the name:
  \let\gplgaddtomacro\g@addto@macro
  % define empty templates for all commands taking text:
  \gdef\gplbacktext{}%
  \gdef\gplfronttext{}%
  \makeatother
  \ifGPblacktext
    % no textcolor at all
    \def\colorrgb#1{}%
    \def\colorgray#1{}%
  \else
    % gray or color?
    \ifGPcolor
      \def\colorrgb#1{\color[rgb]{#1}}%
      \def\colorgray#1{\color[gray]{#1}}%
      \expandafter\def\csname LTw\endcsname{\color{white}}%
      \expandafter\def\csname LTb\endcsname{\color{black}}%
      \expandafter\def\csname LTa\endcsname{\color{black}}%
      \expandafter\def\csname LT0\endcsname{\color[rgb]{1,0,0}}%
      \expandafter\def\csname LT1\endcsname{\color[rgb]{0,1,0}}%
      \expandafter\def\csname LT2\endcsname{\color[rgb]{0,0,1}}%
      \expandafter\def\csname LT3\endcsname{\color[rgb]{1,0,1}}%
      \expandafter\def\csname LT4\endcsname{\color[rgb]{0,1,1}}%
      \expandafter\def\csname LT5\endcsname{\color[rgb]{1,1,0}}%
      \expandafter\def\csname LT6\endcsname{\color[rgb]{0,0,0}}%
      \expandafter\def\csname LT7\endcsname{\color[rgb]{1,0.3,0}}%
      \expandafter\def\csname LT8\endcsname{\color[rgb]{0.5,0.5,0.5}}%
    \else
      % gray
      \def\colorrgb#1{\color{black}}%
      \def\colorgray#1{\color[gray]{#1}}%
      \expandafter\def\csname LTw\endcsname{\color{white}}%
      \expandafter\def\csname LTb\endcsname{\color{black}}%
      \expandafter\def\csname LTa\endcsname{\color{black}}%
      \expandafter\def\csname LT0\endcsname{\color{black}}%
      \expandafter\def\csname LT1\endcsname{\color{black}}%
      \expandafter\def\csname LT2\endcsname{\color{black}}%
      \expandafter\def\csname LT3\endcsname{\color{black}}%
      \expandafter\def\csname LT4\endcsname{\color{black}}%
      \expandafter\def\csname LT5\endcsname{\color{black}}%
      \expandafter\def\csname LT6\endcsname{\color{black}}%
      \expandafter\def\csname LT7\endcsname{\color{black}}%
      \expandafter\def\csname LT8\endcsname{\color{black}}%
    \fi
  \fi
    \setlength{\unitlength}{0.0500bp}%
    \ifx\gptboxheight\undefined%
      \newlength{\gptboxheight}%
      \newlength{\gptboxwidth}%
      \newsavebox{\gptboxtext}%
    \fi%
    \setlength{\fboxrule}{0.5pt}%
    \setlength{\fboxsep}{1pt}%
\begin{picture}(7200.00,2520.00)%
    \gplgaddtomacro\gplbacktext{%
      \csname LTb\endcsname%%
      \put(528,406){\makebox(0,0)[r]{\strut{}$0$}}%
      \put(528,643){\makebox(0,0)[r]{\strut{}$1\times10^{8}$}}%
      \put(528,880){\makebox(0,0)[r]{\strut{}$2\times10^{8}$}}%
      \put(528,1116){\makebox(0,0)[r]{\strut{}$3\times10^{8}$}}%
      \put(528,1353){\makebox(0,0)[r]{\strut{}$4\times10^{8}$}}%
      \put(528,1590){\makebox(0,0)[r]{\strut{}$5\times10^{8}$}}%
      \put(528,1827){\makebox(0,0)[r]{\strut{}$6\times10^{8}$}}%
      \put(528,2063){\makebox(0,0)[r]{\strut{}$7\times10^{8}$}}%
      \put(528,2300){\makebox(0,0)[r]{\strut{}$8\times10^{8}$}}%
      \put(660,274){\rotatebox{-45}{\makebox(0,0)[l]{\strut{}1}}}%
      \put(753,274){\rotatebox{-45}{\makebox(0,0)[l]{\strut{}2}}}%
      \put(940,274){\rotatebox{-45}{\makebox(0,0)[l]{\strut{}4}}}%
      \put(1313,274){\rotatebox{-45}{\makebox(0,0)[l]{\strut{}8}}}%
      \put(1686,274){\rotatebox{-45}{\makebox(0,0)[l]{\strut{}12}}}%
      \put(2060,274){\rotatebox{-45}{\makebox(0,0)[l]{\strut{}16}}}%
      \put(2806,274){\rotatebox{-45}{\makebox(0,0)[l]{\strut{}24}}}%
      \put(3553,274){\rotatebox{-45}{\makebox(0,0)[l]{\strut{}32}}}%
      \put(5046,274){\rotatebox{-45}{\makebox(0,0)[l]{\strut{}48}}}%
      \put(6539,274){\rotatebox{-45}{\makebox(0,0)[l]{\strut{}64}}}%
    }%
    \gplgaddtomacro\gplfronttext{%
      \csname LTb\endcsname%%
      \put(2904,2127){\makebox(0,0)[r]{\strut{}writes}}%
      \csname LTb\endcsname%%
      \put(2904,1907){\makebox(0,0)[r]{\strut{}writes-array}}%
      \csname LTb\endcsname%%
      \put(2904,1687){\makebox(0,0)[r]{\strut{}writes-and-reads}}%
    }%
    \gplbacktext
    \put(0,0){\includegraphics{sharedintserver}}%
    \gplfronttext
  \end{picture}%
\endgroup

		\caption{xeon}
	\end{subfigure}
	\caption{Contended increments on \textbf{non-volatile} integers. The
	plots show nanoseconds per increment, as a function of the number of
	threads. Please note that both axes vary across the plots.}
	\label{fig:cont-nob}
\end{figure}

\mynote{include plots as ns-per-operation, concrete numbers/factors, and
discussion/conclusion on the results, concrete parameters: work not divided..}

\mynote{Konklusioner: Læsning er ikke gratis(!), pris for contended writes (både
med og uden barrier/volatile) siger en del om MESI pris (1 vs 2 tråde er
illustrativt), indikerer (løst) pris af at smide en cache-line frem og tilbage
mellem kerner}


\subsection{Practical Applications}
While incrementing counters is useful, it is thankfully not the only thing we
use software for. It is therefore useful to see the effects of false sharing in
more complicated applications as well.

In this section we examine false sharing in 4 \mynote{Update if it doesn't
end up 4} parallel programs. We use Quicksort as an example of the divide and
conquer paradigm, where the division of subproblems limit the need for
synchronization. We take k-means clustering and a striped hashmap implementation
as examples of locking applications. Finally, we \mynote{more problems go here
when found}

\subsubsection{Divide and Conquer \mynote{avoidance? Quicksort}}
A famous example of a divide-and-conquer algorithm, quicksort recursively
divides the sorting problem into independent subproblems, combining partial
results into a solution for the whole problem. With respect to synchronization,
we can see this problem division as a kind of avoidance: If the sub-problems are
independent, they can be solved in parallel on a multicore system without any
communication - and therefore synchronization - between parallel processes.
Of course, we need to ensure that solutions are visible after they are
found.

\mynote{quicksort insert code snippet}

The program is a (somewhat naive) parallel implementation of quicksort. Anytime
the algorithm divides the problem in two, the calling thread forks a task to
solve the left part of the problem, executing the right part itself. Forked
tasks are run in parallel at the discretion of the java \java{ForkJoinPool}.

As is common, and as is recommended in the popular algorithms text book by
\citet{sedgewick}, the implementation uses a cutoff to a sequential selection
sort. An experiment is included to find a good value for the cutoff.

There is no step to combine the results of two subproblems, as calls operate
in-place on the same shared array. While this may \textit{seem} like communication
between threads, there is only a single thread operating on a given
array-segment at any time. The only relevant communication is that of the
constructor parameters to the recursive calls, and the array. Operations in
recursive calls are guaranteed to be visible to the caller because of the
call to \java{join()}.

\mynote{include concrete parameters: list size(s), cutoff details}
\mynote{Analyze/conclude}

\subsubsection{Locking \mynote{kmeans, striped hashmap}}
As examples of locking applications, we examine an implementation of k-means
clustering, as suggested in the January 2017 exam in Practical Concurrent and
Parallel Programming (PCPP) at the IT-Univeristy of Copenhagen \cite{kmeansexam,
kmeansexamcode}, As well as implementation of a concurrent hashmap datastructure
used in the same course. Both of these applications are studied in
\cite{mystery}, and can be said to have instigated this project.

\mynote{subdivide eg. with subsubsuectiont, and complete kmeans and explanations}

\subsubsection{Lock free\mynote{(? cas-based?)}}
\begin{figure}[hbpt]
	\graphicspath{{plots/}}
	\begin{subfigure}{0.32\textwidth}
		\input{plots/Histo-lock-local-0-x1.tex}
		\caption{0 bytes}
	\end{subfigure}
	\begin{subfigure}{0.32\textwidth}
		\input{plots/Histo-lock-local-16-x1.tex}
		\caption{16 bytes}
	\end{subfigure}
	\begin{subfigure}{0.32\textwidth}
		\input{plots/Histo-lock-local-32-x1.tex}
		\caption{32 bytes}
	\end{subfigure}
	\begin{subfigure}{0.32\textwidth}
		\input{plots/Histo-lock-local-48-x1.tex}
		\caption{48 bytes}
	\end{subfigure}
	\begin{subfigure}{0.32\textwidth}
		\input{plots/Histo-lock-local-64-x1.tex}
		\caption{64 bytes}
	\end{subfigure}
	\begin{subfigure}{0.32\textwidth}
		\input{plots/Histo-lock-local-80-x1.tex}
		\caption{80 bytes}
	\end{subfigure}
	\begin{subfigure}{0.32\textwidth}
		\input{plots/Histo-lock-local-96-x1.tex}
		\caption{96 bytes}
	\end{subfigure}
	\begin{subfigure}{0.32\textwidth}
		\input{plots/Histo-lock-local-112-x1.tex}
		\caption{112 bytes}
	\end{subfigure}
	\begin{subfigure}{0.32\textwidth}
		\input{plots/Histo-lock-local-128-x1.tex}
		\caption{128 bytes}
	\end{subfigure}
	\caption{The histogram problem on the i5 platform. Here the impact of
	padding the locks is negative, while padding the buckets improves
	performance by 10.7\%. The best time is 100.88 milliseconds, achieved by
	padding the buckets with 128 bytes, and not padding the locks.
	Each plot uses a different amount of padding between the locks
	(specified beneath each plot). The plots show the running time of the
	histogram problem as a function of the amount of padding between the
	histogram buckets in bytes. Each y-tick is 50 milliseconds. The axis
	starts at 0.}
	\label{fig:histo-local-i5}
\end{figure}

\begin{figure}[hbpt]
	\graphicspath{{plots/}}
	\begin{subfigure}{0.32\textwidth}
		\input{plots/Histo-lock-local-0-itu-desktop.tex}
		\caption{0 bytes}
	\end{subfigure}
	\begin{subfigure}{0.32\textwidth}
		\input{plots/Histo-lock-local-16-itu-desktop.tex}
		\caption{16 bytes}
	\end{subfigure}
	\begin{subfigure}{0.32\textwidth}
		\input{plots/Histo-lock-local-32-itu-desktop.tex}
		\caption{32 bytes}
	\end{subfigure}
	\begin{subfigure}{0.32\textwidth}
		\input{plots/Histo-lock-local-48-itu-desktop.tex}
		\caption{48 bytes}
	\end{subfigure}
	\begin{subfigure}{0.32\textwidth}
		\input{plots/Histo-lock-local-64-itu-desktop.tex}
		\caption{64 bytes}
	\end{subfigure}
	\begin{subfigure}{0.32\textwidth}
		\input{plots/Histo-lock-local-80-itu-desktop.tex}
		\caption{80 bytes}
	\end{subfigure}
	\begin{subfigure}{0.32\textwidth}
		\input{plots/Histo-lock-local-96-itu-desktop.tex}
		\caption{96 bytes}
	\end{subfigure}
	\begin{subfigure}{0.32\textwidth}
		\input{plots/Histo-lock-local-112-itu-desktop.tex}
		\caption{112 bytes}
	\end{subfigure}
	\begin{subfigure}{0.32\textwidth}
		\input{plots/Histo-lock-local-128-itu-desktop.tex}
		\caption{128 bytes}
	\end{subfigure}
	\caption{The histogram problem on the i7 platform. Here the overall
	trend is that padding improves performance with respect to both buckets
	and locks. The best time is 59.9 milliseconds, achieved by padding the
	locks with 112 bytes, and the buckets with 128 bytes. That is a 16.9\%
	improvement on not using padding. Each plot uses a different amount of
	padding between the locks (specified beneath each plot). The plots show
	the running time of the histogram problem as a function of the amount of
	padding between the histogram buckets in bytes. Each y-tick is 10
	milliseconds. The axis starts at 0.} \label{fig:histo-local-i7}
\end{figure}

\begin{figure}[hbpt]
	\graphicspath{{plots/}}
	\begin{subfigure}{0.32\textwidth}
		\input{plots/Histo-lock-local-0-itu-server.tex}
		\caption{0 bytes}
	\end{subfigure}
	\begin{subfigure}{0.32\textwidth}
		\input{plots/Histo-lock-local-16-itu-server.tex}
		\caption{16 bytes}
	\end{subfigure}
	\begin{subfigure}{0.32\textwidth}
		\input{plots/Histo-lock-local-32-itu-server.tex}
		\caption{32 bytes}
	\end{subfigure}
	\begin{subfigure}{0.32\textwidth}
		\input{plots/Histo-lock-local-48-itu-server.tex}
		\caption{48 bytes}
	\end{subfigure}
	\begin{subfigure}{0.32\textwidth}
		\input{plots/Histo-lock-local-64-itu-server.tex}
		\caption{64 bytes}
	\end{subfigure}
	\begin{subfigure}{0.32\textwidth}
		\input{plots/Histo-lock-local-80-itu-server.tex}
		\caption{80 bytes}
	\end{subfigure}
	\begin{subfigure}{0.32\textwidth}
		\input{plots/Histo-lock-local-96-itu-server.tex}
		\caption{96 bytes}
	\end{subfigure}
	\begin{subfigure}{0.32\textwidth}
		\input{plots/Histo-lock-local-112-itu-server.tex}
		\caption{112 bytes}
	\end{subfigure}
	\begin{subfigure}{0.32\textwidth}
		\input{plots/Histo-lock-local-128-itu-server.tex}
		\caption{128 bytes}
	\end{subfigure}
	\caption{The histogram problem on the Xeon platform. Here padding has a
	large effect, both for the buckets and the locks. The best time is
	118.62 miliseconds, achieved by padding the locks with 112 bytes, and
	the histograms with 128 bytes. That is a 71.8\% improvement on not using
	padding.
	Each plot uses a
	different amount of padding between the locks (specified beneath each
	plot). The plots show the running time of the histogram problem as a
	function of the amount of padding between the histogram buckets in
	bytes. Each y-tick is 100 milliseconds. The axis starts at 0.}
	\label{fig:histo-local-xeon}
\end{figure}

\mynote{this needs to be done}
